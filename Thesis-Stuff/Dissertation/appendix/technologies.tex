\section{{Quimera Implementation}} \label{ss:tech}

\quim\ system follows a typical web-application schema: a server-client framework connects users to a server, where submissions and system data are recorded.
\quim\ User Interface is rendered in \textsf{HTML 5}, \textsf{CSS 3} and \textsf{Javascript}, optimized for the last generation browsers but also compatible with older browsers without loosing any features.
%It also provides some client-side animations to improve experience, for this \quim\ uses \textsf{MooTools}\footnote{\url{http://mootools.net/}}, a javascript object-oriented framework.

The system runs on any \textsf{PHP 5.3-enabled} platform over \textsf{HTTP} and \textsf{HTTPS} protocols, and it is based on the \textsf{Symfony2}\footnote{\url{http://symfony.com}} framework.
The essential application data is stored in a \textsf{MySQL} database, although the data layer of \textsf{Symfony2} (managed by \textsf{Doctrine2} technology) allows to use any other database engine.
\textsf{Twig} template manager, embedded in this framework, is used to render HTML5 templates from the view layer.

%To do the source code analysis, we developed a service from \textsf{Symfony2}, to enable expansibility.
%Now we describe its modules.
To compile the submitted solutions, the system uses the \textsf{GCC} compiler.
For the static analysis, it was developed a plugin for \textsf{Frama-C}\footnote{\url{http://frama-c.com}}, a framework dedicated to the static analysis of source code written in C.
This powerful platform works as a front-end for the plugin, once its kernel provides common functionalities, libraries and collaborative data structures, which simplifyies the development process and improves the final results.
\textsf{CIL}\footnote{C Intermediate Language \url{http://cil.sourceforge.net/}} library provides methods to  generate the \textsf{Abstract Syntax Tree (AST)} for the source code and to traverse the tree.
The plugin analyzes the AST for each submitted source file and evaluates a set of metrics (described in section~\ref{ss:metrics}) with a single traversal, to obtain quantitative information about the  quality of the  source code.

%For the dynamic analysis, each submitted program is executed over a set of input data; the obtained output is then compared with the  set of output values expected for the problem being solved.

Quimera's Cloning detection is based on regular expressions, string comparisons and sorting algorithms from PHP, to detect duplicated lines of code in the submission.

In order to detect plagiarism, it was used the tool \textsf{sherlock}\footnote{\url{http://sydney.edu.au/engineering/it/~scilect/sherlock/}}, which makes comparisons based on source code signatures, and compare them with related submitted code.
This tool supports its comparison algorithm in a \emph{Originality Index} that computes the ratio between similarities and differences detected in each pair of files.
This plugin produces a final result that contemplates all these evaluation parameters as well as the final grade assessed by the system, as described in section~\ref{ss:formula}.

\quim\ provides several statistical graphics related with a contest flow (as described in section~\ref{s:stats}), rendered with \textsf{Google Chart Tools\footnote{\url{http://code.google.com/intl/en-US/apis/chart/}}}.





