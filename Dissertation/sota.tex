\chapter{State of the Art}\label{c:stateofArt}

This chapter aims to present previous developed work regarding the main concepts on what this dissertation rests (Section \ref{c:relWork}), which is then divided into different subsections. The subsection \ref{s:relWork-sec} aims to provide a comprehensive overview over works that attended to prevent security issues related to the ROS architecture. %deployment of robotic systems, using Robot Operating System as its application enhancer. 
It is then followed by the subsection % \ref{s:relWork-sa} regarding some of the current static analysis approaches on ROS applications. The latter subsection 
\ref{s:relWork-pv} that concerns about previous work addressing property verification and model checking techniques over the robotics field of exploration.

Afterwards, the section \ref{c:currWork} provides a brief introduction over the problems that this dissertation aims to tackle, supported by a graphical dissertation schedule, in which the main development tasks are organized over time.

\section{Related work}\label{c:relWork}

As reliance on robotic systems grows due to their expanding application across a wide range of domains, these systems concern critical scenarios, where human interaction comes in hand \cite{diluoffo2018robot}. Thus, security should be highly considered upon employing systems that might jeopardize the human's integrity, such as robotics. Additionally, ensuring analysis regarding the system's quality assurance is increasingly becoming a focus of attention. Thus, it is critical to employ techniques that promote the increase of the system quality, without sacrificing its flexible nature.

This section concerns the study of some relevant studies regarding the main topics addressed by the current dissertation. Even though this dissertation was devoted to a detailed examination of ROS framework regarding its security deployment, there is no obvious line separating the ROS difficulties from those of other popular multi-configurable robotic software in terms of quality assurance and security overview. As result, a broader range of robotic systems will be explored to give contextualization of some relevant aspects that also fall under the ROS domain.

As previously noted, the first presented subsection \ref{s:relWork-sec} confers a notable overview over ROS security environment, that will focus on the following topics: It is intended to begin by presenting some works that demonstrate the absence of security over the ROS environment, that is then followed by some attempts regarding solutions; At last, ROS2-based studies will be given to contextualize the dissertation's main topic of study. 

It is then followed by subsection \ref{s:relWork-pv} that introduces to some pertinent quality assurance procedures over the robotics domain. The latter's concentration will be on the following subjects: First, system analysis regarding static procedures will be contextualized, emphasizing a notable framework, called \textit{HAROS}, that performs static evaluation over ROS-based applications; After that, various state-of-art methodologies based on property verification are given as support to this dissertation analysis approach. 


\subsection{Security Overview}\label{s:relWork-sec}

Network security entails cautious analysis of the system's network using realistic networking security procedures \cite{marin2005network}. Concerning the Robot Operating System and its role as a robotic application enhancer, numerous researchers have examined the usage of such procedures to perform a thorough analysis over the latter's architecture.

\citeauthor*{8794451} presents a practical overview over ROS security, in which the \textit{IPv4 address space of the Internet} is explored with the goal of identifying vulnerable hosts. Port scanning was used as technique to expose mostly master nodes as they provide valuable information about their associated topics and node's parameters. The performed scans furnished information about hosts that could either be a sensor, an actuator or even a simulator. This study is rather relevant because of how easily can attackers gather information about potential robots, and control them further on, through the public Internet, making it unavoidable to develop mechanisms concerning security. 

Moreover, in \citenum{dieber2020penetration}, it is presented different pentesting tools that entails exploiting techniques over ROS-based systems, in order to provide a proper overview of possible security flaws. Foremost, \citeauthor{dieber2020penetration} presents a \textit{.net-based pentesting} tool called \textit{ROSPenTo}, developed with the intention of investigating strategies for manipulating running ROS applications. The latter provides several command line techniques to jeopardize robotic networks, including the ROS parameter server, that confers great value to the running nodes. The other tool is called \textit{ROSchaos} wittingly designed for exploiting the Master API. It differs from the former tool, since \textit{ROSPenTo} mainly focused on exploiting ROS \textit{Slave API}, which covers the node-to-node communication and the receipt of messages from the Master, without directly addressing the ROS Master as a compelling target. Regardless matter how subtle such attacks are, exploiting the Master directly may still be appealing to attackers. 

These techniques confer great value to the domain of security exploration over ROS, where the \textit{XML-RPC} embedded API is divided and subsequently exploited according to the participants roles within the network. Besides raising awareness of the importance of security in ROS, it promotes developers to conduct penetration testing on their applications.

Following the challenges that arose as a result of executing exploitation techniques on the ROS framework, numerous solutions were proposed in response to the need to offer security assurance for robotic applications. 

In \citenum{white2016sros}, \citeauthor{white2016sros} addresses the security deployment over ROS, by proposing the \textit{Secure Robot Operating System} (SROS) as a planned enhancement to the former API, that includes mechanisms such as authentication, access control and cryptography measures. It was initially developed as an experimental tool, which supports TLS for all socket transport, node restrictions and chains of trust. \citenum{white2018procedurally} \citenum{white2019network}

Additionally, in \citenum{application-security-ros} it is presented a fairly pertinent research that recommends security improvements oh the application-layer. The approach primarily focused on applying security measures on the application layer, by mainly running an Authentication Server, storing certificates and files related to trusted domain participants, while controlling and providing session keys related to the communication process. Despite encryption and authentication mechanisms, attacks regarding the exposure of the network, such as denial of service attacks, still persist which cannot be handled on the application level alone.


Another worth-mentioned research is presented in \citenum{crick2017rosbridge}. The latter describes the \textit{rosbridge} middleware that adds to the former ROS architecture an abstraction layer. It provides a socket-based interface through a technology standard, where minimalist services are accessible to developers. The research also regards the application of \textit{rosbridge} over different domains.

is \textit{Rosbridge}, presented in \citenum{crick2017rosbridge}, which provides a \textit{WebSocket} interface to the framework and corresponding server to allow interaction between applications and ROS nodes, by using TLS as support and also access control over topics and API calls. 

A study that is rather relevant because of the similarity between their proposal and the one that SROS2 has to offer is the one presented on the \textit{Application-level security for ROS-based applications} \cite{application-security-ros}. The approach primarily focused on applying security measures on the application layer, by mainly running an Authentication Server, storing certificates and files related to trusted domain participants, while controlling and providing session keys related to the communication process. Even though encryption and authentication measures are concerned, the protected network is still perceived from the "outside", meaning that security attacks, such as denial of service, still persist which cannot be handled on the application level alone. Secure Robot Operating System (SROS) \cite{white2016sros} was initially developed as an experimental tool (later evolved to SROS2 as a supporting tool for ROS2), which supports TLS for all socket transport, node restrictions and chains of trust, guaranteeing publishers authorization when it comes to publish to a specific topic. Another worth-mentioned tool is Rosbridge \cite{crick2017rosbridge}, which provides a WebSocket interface to ROS and corresponding server to allow interaction between applications and ROS nodes, by using TLS as support and also access control over topics and API calls. 


The literature concerning the network security enhancement that Robot Operating System 2 furnishes, by offering the SROS2 toolset, is quite limited. Most of the existing work is on the exploration of the former version of ROS in terms of port exposure, contextualized in the approach considered to protect the system network.  


The present works addressing ROS security methods tend to concern solutions to prevent vulnerabilites and issues that might compromise robotic applications deployed, while considering performance as priority. In terms of applying formal methods to verify properties regarding the domain of ROS2 and ROS2 security as models, there are minimal existent works. Despite this, the following section consists of several articles proposed to validate robotic systems, using formal methods as core.

\subsection{Analysis and Verification}\label{s:relWork-pv}

Static analysis over the ROS framework represents a major contribution to this dissertation domain, in which researchers aim to tackle issues arose from miss configurations or code inconsistencies. 

\subsubsection{\textit{HAROS} Framework}

The noteworthy \textit{HAROS} framework \cite{santos2016framework} holds great value thanks to its contribution on improving ROS's software quality. \textit{HAROS} makes use of several analysis techniques to exert quality evaluation of ROS software, followed by ways of feedbacking inconsistencies using predefined code metrics. As this framework seeks to be flexible when it comes to adding functionality, further static analysis works improvements have been proposed as plugins. In both \citenum{santos2019static} and \citenum{santos2018property}, it is presented additional functionality to the framework, through applying architectural considerations over metamodel designing, where the latter supports the former by supplying property-based specifications. These techniques confers great help back to developers, since static analysis offers advantageable usage over raw review of software code. 

The literature concerning property verification over model checking tools is quite extensive. Regarding ROS applications, some approaches were presented that mainly focused on modelling the ROS node-communication, while real-time properties were also considered as support to the target language. In \citenum{halder2017formal}, \textit{UPPAAL} model checker is used to model ROS applications, supported by a concrete robot example, that is followed by techniques to verify properties regarding its behaviour. In \citenum{9341085} is presented a notable proposal, where \textit{Electrum} \cite{lwspecification}, the former version of the current Alloy Analyzer, is used as an additional plug-in to the already mentioned \textit{HAROS} framework. Through ROS launch configurations, the plugin automatically generated models using Electrum and performs verification over these models, to then feedback issues related to their ROS system behaviour.

As ROS2 domain regards the use of DDS communication protocol, a few works on DDS modelling analysis deserve to be mentioned, as they might give important background for property verification over communications protocols. In \citenum{alaerjan2017modeling}, it is proposed a technique to model the DCPS architectural design that DDS makes use of, alongside with new approaches to the current DDS behaviour. Supported by several modelling techniques for publish-subscribe systems, in \citenum{liu2018formal}, DDS in ROS2 is formalized as a timed automata, consequently followed by model verification over property-checking. These works conceive value concepts and procedures useful for this dissertation contextualization.

A few studies on robotics should be recognized in which techniques such as model-checking were performed. Despite the fact that they do not address the domain of ROS, they nonetheless give helpful background for the robotics research over formal approaches. A case study, mentioned in \citenum{near2011lightweight}, presents a novel approach over systems that require static analysis based on software assumptions and proper analysis within its environment usage, where user-interaction comes in hand. Concerning this novel idea, a medical system is concerned as a case study, where multiple safety-based considerations are expected, as well as, an end-to-end critical property that must be satisfied over the entire analysis course. Another notable work regarding former analysis using Alloy specification language is presented in \citenum{mansoor2018modeling}, where a safety-critical scenario is proposed under the domain of surgical robots. The formal techniques used allows overview over a surgical robot arm, taking into consideration possible violations of important safety properties. Although these studies presents favorable outcomes, their focus lies on a particular area of study. As a result, they lack on providing solutions to a vast majority of situations.

Despite the presented work... 

\section{Current Work}\label{c:currWork}