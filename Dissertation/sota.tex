\chapter{State of the Art}\label{c:stateofArt}

This chapter aims to present previous developed work regarding the main concepts on what this dissertation rests (Section \ref{c:relWork}), which is then divided into different subsections. The subsection \ref{s:relWork-sec} aims to provide a comprehensive overview over works that attended to prevent security issues related to the ROS architecture. %deployment of robotic systems, using Robot Operating System as its application enhancer. 
It is then followed by the subsection % \ref{s:relWork-sa} regarding some of the current static analysis approaches on ROS applications. The latter subsection 
\ref{s:relWork-pv} that concerns about previous work addressing property verification and model checking techniques over the robotics field of exploration.

Afterwards, the section \ref{c:currWork} provides a brief introduction over the problems that this dissertation aims to tackle, supported by a graphical dissertation schedule, in which the main development tasks are organized over time.

\section{Related work}\label{c:relWork}

As reliance on robotic systems grows due to their expanding application across a wide range of domains, these systems concern critical scenarios, where human interaction comes in hand \cite{diluoffo2018robot}. Thus, security should be highly considered upon employing systems that might jeopardize the human's integrity, such as robotics. Additionally, ensuring analysis regarding the system's quality assurance is increasingly becoming a focus of attention. Consequently, it is critical to employ techniques that promote the increase of the system quality, without sacrificing its flexible nature.

This section concerns the study of some relevant studies regarding the main topics addressed by the current dissertation. Even though this dissertation was devoted to a detailed examination of ROS framework regarding its security deployment, there is no obvious line separating the ROS difficulties from those of other popular multi-configurable robotic software in terms of quality assurance and security overview. As result, a broader range of robotic systems will be explored to give contextualization of some relevant aspects that also fall under the ROS domain.

As previously noted, the first presented subsection \ref{s:relWork-sec} confers a notable overview over ROS security environment, that will focus on the following topics: It is intended to begin by presenting some works that demonstrate the absence of security over the ROS environment, that is then followed by some attempts regarding solutions; At last, ROS2-based studies will be given to contextualize the dissertation's main topic of study. 

It is then followed by subsection \ref{s:relWork-pv} that introduces to some pertinent quality assurance procedures over the robotics domain. The latter's concentration will be on the following subjects: First, system analysis regarding static procedures will be contextualized, emphasizing a notable framework, called \textit{HAROS}, that performs static evaluation over ROS-based applications; After that, various state-of-art methodologies based on property verification are given as support to this dissertation analysis approach. 


\subsection{Security Analysis}\label{s:relWork-sec}

Robotic systems were initially conceived as augmented computers with no explicit boundaries or limitations. As a result of the requirement to provide practical systems as fast as possible, security matters were disregard \cite{white2018procedurally}. Network security entails cautious analysis of the system's network using realistic networking security procedures \cite{marin2005network}. Concerning the Robot Operating System and its role as a robotic application enhancer, numerous researchers have examined the usage of such procedures to perform a thorough analysis over the latter's architecture.

\citeauthor*{8794451} presents a practical overview over ROS security, in which the \textit{IPv4 address space of the Internet} is explored with the goal of identifying vulnerable hosts. Port scanning was used as technique to expose mostly master nodes as they provide valuable information about their associated topics and node's parameters. The performed scans furnished information about hosts that could either be a sensor, an actuator or even a simulator. This study is rather relevant because of how easily can attackers gather information about potential robots, and control them further on, through the public Internet, making it unavoidable to develop mechanisms concerning security. 

Moreover, in \citenum{dieber2020penetration}, it is presented different pentesting tools that entails exploiting techniques over ROS-based systems, in order to provide a proper overview of possible security flaws. Foremost, \citeauthor{dieber2020penetration} presents a \textit{.net-based pentesting} tool called \textit{ROSPenTo}, developed with the intention of investigating strategies for manipulating running ROS applications. The latter provides several command line techniques to jeopardize robotic networks, including the ROS parameter server, that confers great value to the running nodes. 

The other tool is called \textit{ROSchaos} wittingly designed for exploiting the Master API. It differs from the former tool, since \textit{ROSPenTo} mainly focused on exploiting ROS \textit{Slave API}, which covers the node-to-node communication and the receipt of messages from the Master, without directly addressing the ROS Master as a compelling target. Regardless matter how subtle such attacks are, exploiting the Master directly may still be appealing to attackers. 

These techniques confer great value to the domain of security exploration over ROS, where the \textit{XML-RPC} embedded API is divided and subsequently exploited according to the participants roles within the network. Besides raising awareness of the importance of security in ROS, it promotes developers to conduct penetration testing on their applications.

Following the challenges that arose as a result of executing exploitation techniques on the ROS framework, numerous solutions were proposed in response to the need to offer security assurance for robotic applications. 

One of the earliest research on the security improvement of the ROS framework was proposed by \citeauthor{mcclean2013preliminary} in \citenum{mcclean2013preliminary}. Here, the first take is to exploit and reason about unique vulnerabilities related to the nature of cyber-physical systems, where it is demanding to collect data for further consideration. Afterwards, it is presented a novel research tool to aid in cyber-physical security research, named as \textit{honeypot}, where it is desirable to monitor unauthorized attempts to jeopardize these systems. Due to the significant information offered, the latter gives a basic yet crucial study on ROS, which acts as assistance to steer the development of future work in the burgeoning subject of cyber-physical security. 

% Another worth-mentioned research is presented in \citenum{crick2017rosbridge}. The latter describes the \textit{rosbridge} middleware that adds to the former ROS architecture an abstraction layer. It provides a socket-based interface through a technology standard, where minimalist services are accessible to developers.

In \citenum{white2016sros}, \citeauthor{white2016sros} addresses the security deployment over ROS in a more concise way, by proposing the \textit{Secure Robot Operating System} (SROS) as a planned enhancement to the former API, that includes mechanisms such as authentication, access control and cryptography measures. SROS seeks to provide adequate security architecture while minimizing existing user disturbances such as computational cost and API breaking. The final remarks regard the alternative usage of SROS, allowing developers to customize it to their own internal certificate formats. Moreover, \citeauthor{white2016sros} expects security logging and access control to evolve through well-defined standards, enabling more robust auditing and policy generating tools. 

Additionally, in \citenum{application-security-ros} it is presented a fairly pertinent research that recommends security improvements on the application-layer. Briefly speaking, the approach primarily focused on applying security measures on the application layer, by mainly running an Authentication Server, storing certificates and files related to trusted domain participants, while controlling and providing session keys related to the communication process. Topic-specific encryption keys are employed to protect data secrecy. However, the discussed architecture is based on the assumption that issuing authentication certificates are manually handled. Despite encryption and authentication mechanisms, attacks regarding the exposure of the network, such as denial of service attacks, still persist which cannot be handled on the application level alone. 

\citeauthor*{breiling2017secure} continues to follow the previous work in \citenum{breiling2017secure}, through a hardened ROS core with transparent authentication, authorization, and encryption functionalities that do not require the manual specification of nodes. Furthermore, secure workflows and initial penetration testing are supported in \citenum{dieber2017security}, in which \citeauthor{dieber2017security} shows an insight of security weaknesses in the ROS architecture design, with additional attempts that regard potential solutions to those problems.

Regarding the SROS \cite{white2016sros} initial proposal, \citeauthor{white2018procedurally} continued to provide security overview over the framework through subsequent studies \cite{white2018procedurally, white2019network}. In \citenum{caiazza2019enhancing}, \citeauthor*{caiazza2019enhancing} presents several solutions regarding the lack of security measures that autonomous devices, often related to the robotics world, tend to face. The effort then moves on to evaluate the ROS framework in order to provide a high-quality understanding solution. It follows a pipeline of security measurements, where logging, access control and authentication certificates are discussed over several proposals. It provides a clear analysis over the security state-of-art of ROS, and the analysis clearly states that the lack of techniques to prevent communication threats imposes the most valued consideration over robotic networks. \citeauthor*{caiazza2019enhancing} conclude their work by stating some future improvements to their current solution. 

In \citenum{white2018procedurally}, \citeauthor*{white2018procedurally} addresses robotics security through a proposed framework that focuses on handling access control policies for robotic software, with the intention of adding functionality to SROS \cite{white2016sros}. The latter offers an interesting perspective on leveraging access-control through a well-typed markup language schema, the \textit{ComArmor} configuration language, where privileges of objects is explicitly defined over policies. Then, profiles are used to arrange these policies in a hierarchical manner, binding namespaces to objects privileges, utilizing attachment expressions with predefined permissions denoted as rules. Rules are defined as a set of permissions for a given object, where each permission has a corresponding value that could either be allow or deny. 

The evaluation of this policy schema as a proof of concept was done by implementing it in a cryptographic framework called \textit{Keymint}. The latter follows similar authentication and cryptographic patterns as the initial SROS \cite{white2016sros}, where security artifacts and document keys are stored in respective keystores. The incorporation of the ComArmor language into Keymint adds significant value to the framework since it allows for reasoning about how to manage policy control prior to implementing any authentication mechanism, such as generating permission and governance files. The framework is then tested using a standard ROS example of a basic publisher and simple subscriber interacting over a topic. Thus, this approach provides a methodical testing procedure where both application satisfiability and access control policies implementation are overviewed, adding significant value back to the framework as it is possible to define higher level policies to allow for transport-independent access control definitions.

The \textit{ComArmor} configuration language, presented above in \citenum{white2018procedurally}, confers a great alternative to the SROS2 default policy schema. During the framework's evaluation process, \citeauthor{white2018procedurally} presents a set of gratuitous permissions within SROS2 template through a simple comparative study between both configuration languages. Thus, it discusses possible SROS2 vulnerabilities discovered during the development and experimentation the \textit{Keymint} framework, emphasizing the importance of continuous security evaluation throughout design development cycle.

Following the deployment of DDS in Robot Operating System 2, the majority of security problems were alleviated. As a result, the literature on the network security enhancements provided by ROS2 is largely concerned with providing an overview of the trade-off between performance and security. Furthermore, various inquiry studies evaluate ROS2 performance in terms of response time and safety important circumstances \cite{maruyama2016exploring, casini2019response}, in which DDS capabilities are highly emphasized to properly evaluate ROS2.

A notable work is presented in \citenum{kim2018security}, where the security definition of DDS was explored using current security standards. Also, the implementation of the DDS protocol was evaluated using techniques of static analysis. It provides a framework for doing benchmark evaluations in various network and security configurations to reflect on the needs of real-time applications. Different performance time-related metrics and benchmark network scenarios (wired and wireless) were thoroughly investigated to give a well-based analysis over efficiency. The findings clearly illustrate that employing a reliable VPN application is more time efficient; nevertheless, SROS offers far more decentralized features than a VPN, which is typically desirable when designing a distributed system. \citeauthor*{kim2018security} conclude their work by reasoning about the security capabilities that DDS has to offer, after which a static code analysis of the DDS security implementation was undertaken. The research revealed that the latter did not conform to the security specfication by OMG. 

A pertinent research that examines the trade-off between performance and security is regarded in \textit{Robot Operating System 2: The need for a holistic security approach to robotic architectures} \cite{diluoffo2018robot}. Here, \citeauthor{diluoffo2018robot} conducts a thorough examination of Robot Operating System 2 and outlines possible hazards for this new robotic system paradigm. Thus, the DDS security specification is thoroughly examined in terms of performance versus security models and how security is integrated with ROS 2, as well as how the addition of security affects system design. 

As a result, \citeauthor{diluoffo2018robot} goes on to present a systematic study of ROS2 in order to provide a good knowledge of how the system is built, and to subsequently undertake vulnerability analysis across various layers that require varying levels of protection. Additionally, the authors applied different performance metrics to duly present a well-based comparasion in whether security should be considered or not. Here, they consider the usage of DDS-based implementations, while addressing the RTPS communication protocol and it handles the communication and its configuration. The experiment illustrates that implementing security measures on protected data in motion results in a significant performance loss. 

Notably, they conclude with some recommendations towards security areas not duly covered by the DDS specfication, such as attacks stemming from
spy processes. They highlighted the preservation of the cognitive layer because of its importance in the robot itself, since all types of sensors that involve data collecting are concerned in that layer. Here, an advanced set of threats were described using Machine Learning, in which DDS is not capable of cover.

These presented works denote important considerations about how important is to address security over robotics, due to the variaty of attacks that might compromise their integrity. They follow the path of addressing vulnerabilities through applying security methods based on tools and protocols. However, studies regarding the appliance of quality analysis over static and property verification in ROS are quite limited. Despite this, the section that follows contains several works that advocate utilizing quality analysis to assess robotic systems.


\subsection{Analysis and Verification}\label{s:relWork-pv}

Analysis through property verification in the ROS framework represents a major contribution to this dissertation domain, in which researchers aim to tackle issues arose from miss configurations or code inconsistencies. Although researchs and tools on analysis of ROS systems are scarce, there are some presented work that might be useful within the context of this dissertation.

For instance, a natural way of performing analysis over a system is by observing the communication architecture. Here, ROS provides a framework for GUI development plugin called \textit{rqt\_graph} \footnote[2]{http://wiki.ros.org/rqt\_graph}, in which ROS developers usually rely on. It is used to depict the network architecture through a graph with run-time statistics. However, this approach lacks on providing trustworthy analysis over more complex networks, where features such as security control are implicit.

A noteworthy study is supplemented by \citeauthor*{white2019network}, in \textit{Network Reconnaissance and Vulnerability Excavation of Secure DDS Systems} \cite{white2019network}.

The noteworthy \textit{HAROS} framework, initially proposed by \citeauthor{santos2016framework} in \textit{A framework for quality assessment of ROS repositories} \cite{santos2016framework}, holds great value thanks to its contribution on improving ROS's software quality. \textit{HAROS} makes use of several analysis techniques to exert quality evaluation of ROS software, followed by ways of feedbacking inconsistencies using predefined code metrics. As this framework seeks to be flexible when it comes to adding functionality, further static analysis works improvements have been proposed as plugins.

In both \citenum{santos2019static} and \citenum{santos2018property}, it is presented additional functionality to the framework, through applying architectural considerations over metamodel designing, where the latter supports the former by supplying property-based specifications. These techniques confers great help back to developers, since static analysis offers advantageable usage over raw review of software code. 

Another relevant study regarding the software verification in robotics is presented by \citeauthor{cortesi2013static} in \textit{Static Analysis Techniques
for Robotics Software Verification} \cite{cortesi2013static}. 

The literature concerning property verification has already been explored through state-of-art model checkers, where safety properties regarding the robotics domain were studied and verified. Within the ROS context, certain techniques were proposed that primarily focused on modeling the ROS node-communication, while real-time properties were also considered as support to the target language.
 
In \citenum{halder2017formal}, \textit{UPPAAL} model checker is used to model ROS applications, supported by a concrete robot example, that is followed by techniques to verify properties regarding its behaviour. 

A remarkable work that duly fits under this dissertation domain of exploration is presented in \citenum{9341085}. Here, \citeauthor{9341085} presents a notable proposal, where \textit{Electrum} \cite{lwspecification}, the former version of the current Alloy Analyzer, is used as an additional plug-in to the already mentioned \textit{HAROS} framework. Through ROS launch configurations, the plugin automatically generated models using Electrum and performs verification over these models, to then feedback issues related to their ROS system behaviour.

As ROS2 domain regards the use of DDS communication protocol, a few works on DDS modelling analysis deserve to be mentioned, as they might give important background for property verification over communications protocols. In \citenum{alaerjan2017modeling}, it is proposed a technique to model the DCPS architectural design that DDS makes use of, alongside with new approaches to the current DDS behaviour. Supported by several modelling techniques for publish-subscribe systems, in \citenum{liu2018formal}, DDS in ROS2 is formalized as a timed automata, consequently followed by model verification over property-checking. These works conceive value concepts and procedures useful for this dissertation contextualization.

A few studies on robotics should be recognized in which techniques such as model-checking were performed. Despite the fact that they do not address the domain of ROS, they nonetheless give helpful background for the robotics research over formal approaches. 

A case study, mentioned in \citenum{near2011lightweight}, presents a novel approach over systems that require static analysis based on software assumptions and proper analysis within its environment usage, where user-interaction comes in hand. Concerning this novel idea, a medical system is concerned as a case study, where multiple safety-based considerations are expected, as well as, an end-to-end critical property that must be satisfied over the entire analysis course. Another notable work regarding former analysis using Alloy specification language is presented in \citenum{mansoor2018modeling}, where a safety-critical scenario is proposed under the domain of surgical robots. The formal techniques used allows overview over a surgical robot arm, taking into consideration possible violations of important safety properties. Although these studies presents favorable outcomes, their focus lies on a particular area of study. As a result, they lack on providing solutions to a vast majority of situations.

\section{Current Work}\label{c:currWork}