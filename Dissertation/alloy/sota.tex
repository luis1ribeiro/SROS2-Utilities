\chapter{Alloy Specification Framework}\label{c:alloy}

As aforementioned, this thesis aims to tackle the security vulnerabilities resulted from the miss-configuration over ROS files. In this chapter, it is intended to explore the Alloy framework that is relevant to overcome the above-mentioned challenge, as well as previous developed work that has the same or similar goals as this thesis (\ref{s:alloy-relWork}).

As result of the increased usage of robotics, and with their integration into the human world, security ensurance for robotics software is highly required. The use of formal methods, especially in systems that require flexibility and reliability, is recommended to avoid security-critical faults. \cite{carvalho2020analysis} Software frameworks designed for this purpose must provide methods to perform structural design over systems with rich structures, abstracting them as a conventional model. Additionally, these frameworks must support features to enable automate analysis, in which property evaluation over these designed models is used as technique. 

The \textit{Alloy Framework} \cite{alloy-DJ}, fits within within this context, as it furnishes a declarative specification language, based on the relation concept, used for software modeling, with extended tools supporting analysis over these models. \cite{alloy-6} The language combination of both relational and linear temporal logic (LTL) enables the ability to model both systems with rich structures and complex behaviour. To address the correctness over the specified model, Alloy performs model-checking techniques over these logic languages, where the latter is exhaustively checked over property verification. \cite{lwspecification, carvalho2020analysis}

The framework includes an IDE, as well as an \textit{analyzer} that takes the specified model's restrictions into account, and performs bounded and unbounded model checking to find instances that satisfy those implied restrictions. It can be also be useful for checking model properties, where the analyzer will try to return a counterexample instance. Instances are displayed by the framework Visualizer, alongisde its divided steps of the modelling process. Instances appearance can be customized, using the \textit{theme}'s extension. \cite{alloy-6}

This chapter will go through these principles in further depth, including how to utilize the Alloy framework, supported by a proper example where ROS communication architecture is structurally modelled.


\section{Structural Design}

\subsection{System Modeling}

\section{Structural Analysis}

\subsection{System Analysis}
% Commands, Instances and Scopes

\subsection{Alloy Analyzer}

\subsection{Model Checking}

\section{Related work}\label{s:alloy-relWork}
