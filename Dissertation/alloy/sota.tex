\chapter{State of the art}\label{c:sota}
As aforementioned, the main problems that this thesis aims to tackle are security vulnerabilities and data processing over big workloads of streaming data. In this chapter, it is intended to explore technologies that are relevant to overcome the above-mentioned challenges (\ref{s:back}) as well as previous developed work that has the same or similar goals as this thesis (\ref{s:relWork}).
\section{Background}\label{s:back}
Enterprises, particularly startups and small and medium businesses, are gradually shifting to the Cloud to outsource data and processing. \cite{6227695} As advantageous as this may be, the sharing of content with third-party systems, poses challenges about privacy and security. As previously stated (\ref{s:problem}), traditional strategies for achieving confidentiality, such as a user encrypting their data with their own key or even using techniques such as homomorphic encryption, make some computations unfeasible, hence alternative methods must be investigated.

% makes either some operations impossible to be performed or still leaves some vulnerabilities opened to be exploited.
% encrypting with his own key still leaves the user exposed when having to perform computations

\subsection{Trusted environments}
\subsubsection{Intel-SGX}
\subsubsection{Gramine}

\subsection{Cryptographic techniques}

\subsection{Apache spark}
\subsubsection{Apache spark structured streaming}


\section{Related work}\label{s:relWork}
